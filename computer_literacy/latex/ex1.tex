\documentclass[a4paper, uplatex]{jlreq}

%パッケージを使おう
\usepackage[dvipdfmx]{graphicx}
\usepackage{comment}

%プリアンブル(オプションの指定。ない場合もある。)
\title{\TeX の使用方法について}
\author{野田慧太朗}
\date{\today}

%マクロ使ってみた
\newcommand{\CL}{コンピューターリテラシー}
\newcommand{\明石}{明石工業高等専門がおっこう電機情報工学科}

\begin{document}
	%maketitle命令(タイトル、著者、日付を出力)
	\maketitle
	
	%目次はtableofcontents
	\tableofcontents
	
	%ページのフォントを指定
	\pagenumbering{roman}	
	
	%(this)pagestyleコマンドでヘッダ、フッダを指定できる。
	\pagestyle{headings}
	
	%ページはpageという変数に格納されているので、setcounter コマンドで変更できる!
	\setcounter{page}{1}
	
	%newpage コマンドでページを更新
	\newpage
	
	\section{はじめに}
	
	これは\TeX の文書の例です。
	
	行を変える時には、このように\textbf{1行空白の行が必要}です。
	数式も書いてみましょう。
	
	\[
	S=\int_a^b f(x) dx
	\]
	
	\section{文書の作り方}
	章や節の番号は自動的に振り分けられる。下に例を示す。
	%\chapter{これは最初の章}
	
	\section{これは最初の節}
	
	\subsection{これは最初の節の最初の行}
	
	\section{これは次の節}
	title, author, date命令で(プリアンブルに書く!)文書のタイトル、著者、日付などを設定して、maketitle命令で出力できるぞ!
	
	(date コマンドはなくしても意味がないので、date{} とする必要がある。)
	
	\newpage
	
	\section{数式を書くぞーーー!}
	\subsection{テキスト用数式モード(インライン数式モード)}
	文章中に$S=a+b$という形で混ざって出力されます。
	
	積分記号$\int_a^bf(x)dx$や和は$\sum_{i=0}^{10} x_i$は行間隔を乱さないように狭く出力されます。
	
	\subsection{ディスプレイ用数式モード}
	\subsubsection{プレーン}
	文章とは独立してこのように
	\[
	\int_a^b f (x) dx
	\]
	出力されます。
	
	\subsubsection{式番あり}
	文章とは独立してこのように
	\begin{equation}
		\int_a^b f (x) dx
	\end{equation}
	出力されます。
	
	\subsubsection{複数行の数式}
	複数行にわたる数式もこの通り。
	\begin{eqnarray}
		F(x) &=& \int_{-\infty}^x f(\zeta)d\zeta\\
		&=& \log |\tan(x)+1|
	\end{eqnarray}
	
	\subsubsection{複数行の数式(式番なし)}
	複数行にわたる数式もこの通り。
	\begin{eqnarray*}
		F(x) &=& \int_{-\infty}^x f(\zeta)d\zeta\\
		&=& \log |\tan(x)+1|
	\end{eqnarray*}
	
	\newpage
	
	\section{表DAZE☆}
	\subsection{表1DAZE☆} 
	\begin{tabular}{|c|l|p{15mm}|} \hline
		月& 別名& よみがな\\ \hline\hline
		1月& 睦月& むつき\\
		2月& 如月& きさらぎ\\
		3月& 弥生& やよい\\
		4月& 卯月& うづき\\ \hline 
	\end{tabular} \\
	\subsection{表2DAZE☆}
	\begin{tabular}{|c|c|c|c|}\hline
		& \multicolumn{2}{|c|}{条件} & \\ \cline{2-3}
		試料& 温度 & 圧力 & 備考 \\ \hline
		1 & 10 & 3 & ちょっと悪い \\ \hline
		1 & 20 & 6 & 結構いい\\
		3 & 40 & 1 & かなり悪い \\ \hline
	\end{tabular} \\
	\subsection{表3DZE☆}
	\begin{table}[http]
		\centering
		\caption{月の別名}
		\label{month-name}
		\begin{tabular}[t]{|c|l|p{15mm}|}\hline
			月& 別名& 読み仮名\\ \hline\hline
			1月& 睦月& むつき\\
			2月& 如月& きさらぎ\\ \hline
		\end{tabular}
	\end{table}
	日本では、表\ref{month-name}に示すように月の別名が定められていて、よく用いられます。
	
	\newpage
	\section{図を入れてみた}
	\subsection{図\ref{nandk1}だドン}
	\begin{figure}[http]
		\centering
		\includegraphics[scale=1]{C:/Users/knoda/OneDrive/ドキュメント/NandK.png}
		\caption{NANDKアイコン}
		\label{nandk1}
	\end{figure}
	
	\newpage
	\subsection{歪んだ図\ref{nandk2}だドン}
	\begin{figure}[http]
		\centering
		\includegraphics[width=100mm, angle=45]{C:/Users/knoda/OneDrive/ドキュメント/NandK.png}
		\caption{歪んだNANDKアイコン}
		\label{nandk2}
	\end{figure}
	
	\newpage
	\section{箇条書きを使ってみた}
	\begin{itemize}
		\item 最初の項目
		\item 次の項目
	\end{itemize}
	
	\begin{enumerate}
		\item 最初の項目
		\begin{enumerate}
			\item 入れ子の最初
			\item 入れ子の二番目
		\end{enumerate}
		\item もとの二番目
	\end{enumerate}
	
	見出し付き箇条書き
	\begin{description}
		\item[見出し] 見出しの説明
		\item[次の見出し] 次の見出しの説明
		\item[見出しが長い場合] 何となく不揃い
	\end{description}
	
	
	
	
\end{document}